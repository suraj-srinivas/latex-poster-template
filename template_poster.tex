\documentclass[paperwidth=76in, paperheight=46in, landscape, margin=5cm, fontscale = 0.215]{baposter}
\usepackage{color, soul}
\definecolor{darkblue}{rgb}{0,0,0.5}
\setulcolor{cyan}

\usepackage{blindtext}
\usepackage[none]{hyphenat}
\usepackage[font=small,labelfont=bf]{caption}
\usepackage{booktabs} % Horizontal rules in tables
\usepackage{relsize} % Used for making text smaller in some places
\usepackage{fancybox}
%% Math
\usepackage{amsmath}
\usepackage{amssymb}
\usepackage{stmaryrd}
\usepackage{eufrak}
\usepackage{layout}
\usepackage{tcolorbox}
\usepackage{graphicx}
\usepackage{subcaption}
\usepackage{array}

%% Algorithms
\usepackage{algorithm}
\usepackage{algpseudocode}
\usepackage{amsmath}
\usepackage{amsthm}
\usepackage{amssymb}
\usepackage{bm}
\usepackage{mathtools}

%% Font
\usepackage{cmbright} % Sans serif in text


%%%%%%%%%%%%%%%%%%%%%%%%%%%%%%%%%%%%%%%%%%%%%%%%%%%%%%%%%%%%%%%%%%%%%%
% Colors
%%%%%%%%%%%%%%%%%%%%%%%%%%%%%%%%%%%%%%%%%%%%%%%%%%%%%%%%%%%%%%%%%%%%%%

% Border color of content boxes
\definecolor{bordercol}{RGB}{255,255,255}

% Background color for the header in the content boxes (left side)
\definecolor{headercol1}{RGB}{255,255,255}

% Background color for the header in the content boxes (right side)
\definecolor{headercol2}{RGB}{255,255,255}

% Text color for the header text in the content boxes
\definecolor{headerfontcol}{RGB}{232,170,12}

% Background color for the content in the content boxes
\definecolor{boxcolor}{RGB}{255,255,255}

\definecolor{col4}{RGB}{116,0,64}

\definecolor{col1}{RGB}{223,84,84}

\definecolor{col3}{RGB}{251,158,63}

\definecolor{col2}{RGB}{142,216,184}


\background{}

%%%%%%%%%%%%%%%%%%%%%%%%%%%%%%%%%%%%%%%%%%%%%%%%%%%%%%%%%%%%%%%%%%%%%%

\begin{document}

\begin{poster}{
grid=false,
columns=4,
eyecatcher=true,
colspacing=1cm,
borderColor=bordercol, % Border color of content boxes
headerColorOne=headercol1, % Background color for the header in the content boxes (left side)
headerColorTwo=headercol2, % Background color for the header in the content boxes (right side)
headerFontColor=headerfontcol, % Text color for the header text in the content boxes
boxColorOne=boxcolor, % Background color for the content in the content boxes
headerfont=\Large, % Font modifiers for the text in the content box headers
textborder=rectangle,
background=user,
headerborder=open, % Change to closed for a line under the content box headers
boxshade=plain}%
{%
 \parbox[c]{0.2\textwidth}{%
%\hspace{1em}

%\hspace{0em}
\includegraphics[height=0.08\textheight]{example-image}
\hspace{30pt}
\includegraphics[height=0.08\textheight]{example-image}

}

}
% Title
{\huge \color{headerfontcol} Your Paper Title} 
% Authors and email
% Authors and email
{ Author1, Author2, Author3, Author 4\\
{ \footnotesize{Hardware University}} }%
 {\parbox[t]{0.18\textwidth}{% 

\raisebox{-0.0cm}{
\centering
\includegraphics[height=0.08\textheight]{example-image}}
\hspace{30pt}
\includegraphics[height=0.08\textheight]{example-image}
}}

\newcommand{\highlight}[1]{%
  \colorbox{red!50}{$\displaystyle#1$}}

\newcommand{\X}{\mathbf{x}}
\newcommand{\Y}{\mathbf{y}}
\newcommand{\W}{\mathbf{w}}
\newcommand{\pmdl}{p_{\theta}}
\newcommand{\grad}{\nabla_{\X}}
\newcommand{\E}{\mathbb{E}}
\newcommand{\f}[1]{\boldsymbol{#1}}
\newcommand{\C}{\mathcal{C}}
\newcommand{\bbR}{\mathbb{R}}
\newcommand{\std}[1]{\tiny{$\pm$ #1}}


%%%%%%%%%%%%%%%%%%%%%%%%%%%%%%%%%%%%%%%%%%%%%%%%%%%%%%%%%%%%%%%%%%%%%%
% Introduction
%%%%%%%%%%%%%%%%%%%%%%%%%%%%%%%%%%%%%%%%%%%%%%%%%%%%%%%%%%%%%%%%%%%%%%

\headerbox{}{name=intro,column=0,row=0,span=1}{
\vspace{1.5cm}
\begin{tcolorbox}
  
Write the main message or the motivation here. Use \underline{underlines} to emphasize key terms. Place your key figure below:

{\begin{minipage}{\textwidth}
  \vspace{0.5cm}
  \centering
  \begin{minipage}{0.4\textwidth}
    \centering
    \includegraphics[width=\linewidth]{example-image}
    {\footnotesize{Caption 1}}
    \label{fig:figure1}
  \end{minipage}%
  \hspace{0.1cm}
  \begin{minipage}{0.4\textwidth}
    \centering 
    \includegraphics[width=\linewidth]{example-image} 
    {\footnotesize{Caption 2}} 
    \label{fig:figure2} 
  \end{minipage}
  \end{minipage}}

\vspace{1cm}
\textbf{Research Question:} How to create a good poster? 

\vspace{1cm}
\textbf{Proposed Approach:}
\begin{itemize}
  \item Mention key takeaways clearly in this box
  \item Don't fill the poster with lots of dense text and equations, except when it is a template (such as this one). Be sure to add lots of pictures
  \item Have fun with the design, and make it your own!
\end{itemize} 

\end{tcolorbox}
}


\headerbox{Section Header}{name=sec1,column=1,row=0,span=1}{

\blindtext

\begin{tcolorbox}
  \vspace*{-10pt}
  \begin{align*}  
  Speed = \frac{distance}{time} {\rightarrow \text{not time} \atop \rightarrow \text{not distance}}
  \end{align*}
\end{tcolorbox}

}

	
\headerbox{Section Header}{name=sec2,column=1,span=1, below=sec1}{

\blindtext

}	


\headerbox{Section Header}{name=sec3,column=2,row=0,span=1}{

\blindtext


}


\headerbox{Section Header}{name=sec4, column=2,span=1, below = sec3}{

\blindtext

}

\headerbox{Section Header}{name=sec5,column=3,row=0,span=1}{

\blindtext

}

\headerbox{Section Header}{name=sec6,column=3,span=1,below=sec5}{

\blindtext

}



\headerbox{References}{name=ack,column=3,span=1, below=sec6}{

[1] Reference 1

[2] Reference 2

[3] Reference 3
}	
	
\end{poster}



\end{document}




